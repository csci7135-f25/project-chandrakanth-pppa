%% This is an abbreviated template from http://www.sigplan.org/Resources/Author/.

\documentclass[acmsmall,review,nonacm]{acmart}
\usepackage{graphicx} % Required for inserting images
\usepackage{mathpartir}
\usepackage{listings}
\let\Bbbk\undefined 
\usepackage{amssymb}
\usepackage{amsmath}
\usepackage{tikz}
\usetikzlibrary{automata, positioning}
\usepackage{algorithm}
\usepackage{algpseudocode}

\newcommand{\I}{\mathbf{I}}
\newcommand{\R}{\mathcal{R}}
\newcommand{\Ll}{\mathbf{L}}
\newcommand{\mathcalX}{\mathcal{X}}
\newcommand{\mathcalY}{\mathcal{Y}}
\newcommand{\mathcalD}{\mathcal{D}}
\newcommand{\setof}[1]{\left\{ #1 \right\}}
\newcommand{\hatx}{\hat{x}} 
\newcommand{\norm}[1]{\left\| #1 \right\|}
\newcommand{\hatJ}{\hat{J}}

\title{Certifying Differential Invariants of Neural Networks using Abstract Duals}
\author{\href{https://ck090.github.io}{Chandra Kanth Nagesh}}
\email{ckn@colorado.edu}
\affiliation{%
  \institution{University of Colorado Boulder}
  \country{USA}
}

\begin{document}

%%
%% The abstract is a short summary of the work to be presented in the
%% article.
\begin{abstract}
  Neural networks are increasingly deployed in safety-critical domains, making the formal verification of their robustness against adversarial perturbations a paramount concern. While relational abstract domains like \texttt{DeepPoly} provide state-of-the-art certification, they suffer from the "wrapping effect," where over-approximation errors accumulate layer-by-layer, potentially failing to verify robust networks. In this work, we propose a complementary verification approach using \textit{Abstract Dual Numbers} to compute global Lipschitz bounds via forward-mode automatic differentiation in the abstract domain. We contribute an algorithmic realization of this method that accounts for curvature error, a verification logic based on gradient bounds, and a "Gradient Instability" metric that serves as a precise predictor of verification failure. Our evaluation identifies a specific regime—shallow networks with smooth activations—where our global gradient method successfully certifies 16.1\% of cases that \texttt{DeepPoly} fails to verify, highlighting the trade-off between local precision and global coherence.
\end{abstract}

%%
%% This command processes the author and affiliation and title
%% information and builds the first part of the formatted document.
\maketitle

\section{Introduction}

\texttt{DeepPoly}\cite{singh_gehr_püschel_vechev_2019} 

\section{Overview}

In this section, we introduce the key insight behind our approach and define the Abstract Dual Domain. Our approach is based on the following intuition: while relational abstractions such as \texttt{DeepPoly}\cite{singh_gehr_püschel_vechev_2019} can be extremely precise, they often suffer from the ``wrapping effect'' where approximation errors accumulate layer by layer. In contrast, global gradient bounds can be sometimes a simpler yet effective verification domain, especially for shallow smooth networks.

\subsection{Paradigm Shift: From Relational Bounds to Gradient Analysis}
The basic underlying concept of this work is to effectively bridge the gap between standard Abstract Interpretation and Forward-Mode Automatic Differentiation. Standard methods like \texttt{DeepPoly} track affine constraints ($x_j \geq \sum w_i x_i$) to bound the output values directly. However, precise reapproximating the feasible set at every non-linear layer introduces error which tends to accumulate with depth ($O(L)$). Our approach, instead, exploits the Mean Value Theorem to find bounds on the output variation in terms of the gradient variation:
\begin{align}
    |f(x) - f(x_0)| \leq \underbrace{\sup_{z \in X} ||\nabla f(z)||}_{\text{Abstract Duals}} \cdot ||x - x_0|
\end{align}
By over-approximating the sound of the gradient $\nabla f(z)$ for the entire input space $X$, we can guarantee robustness because the maximum possible change to the output will be too small to change the classification label. The key insight is that this ``jumps'' over the intermediate layers and thus avoids some of the wrapping errors from the layer-wise value propagation.

\subsection{Abstract Dual Domain}
To formalize this idea, we can start by lifting standard dual numbers to the domain of Affine Arithmetic. Consider the Dual domain $ \hat{\mathcal{D}} $ as a product space of two affine forms representing the range of values and the range of gradients across a set. Then we can say:

\begin{definition}
An \textbf{Abstract Dual Number} $ \mathcal{X} \in \hat{\mathcal{D}} $ is a pair:
\begin{align}
    \mathcal{X} = \langle \hat{x}_{val}, \hat{x}_{grad} \rangle
\end{align}
where:
\begin{itemize}
    \item $ \hat{x}_{val} = \alpha_0 + \sum_{i=1}^n \alpha_i \epsilon_i $ is the affine form representing the interval of neuron values.
    \item $ \hat{x}_{grad} = d_0 + \sum_{i=1}^n d_i \epsilon_i $ is the affine form representing the interval of partial derivatives with respect to the input.
\end{itemize}
\end{definition}

\subsection{Propagation Rules}
To propagate $\mathcal{X}$ through a neural network, we define abstract transformers for each layer type.

\textbf{Linear Layers:} For a fully connected layer with weight matrix $ W $ and bias vector $ b $, the transformation is exactly determined by the linearity of the dual algebra. The affine forms for value and gradient are transformed as follows:
\begin{align}
    \mathcal{Y} = \langle W \hat{x}_{val} + b, W \hat{x}_{grad} \rangle
\end{align}
This operation is exact in the affine domain and introduces no new approximation errors.

\textbf{Non-linear Activations:} For a smooth non-linear activation function $ \sigma $ (e.g., Sigmoid), we cannot directly apply the function to the affine forms. We handle the value and gradient components separately:

\textit{Value Component:} We approximate $ \sigma(\hat{x}_{val}) $ using a linear relaxation. We linearize the function around the center of the input affine form $ c $. The new center is $ \sigma(c) $, and the noise coefficients are scaled by the derivative $ \sigma'(c) $. To ensure soundness, we add a linearization error term to the radius $ r $, which depends on the maximum curvature (second derivative) of $ \sigma $ and the radius of the input interval.
\begin{align}
    \hat{y}_{val} \approx \sigma(c) + \sigma'(c) \cdot (\hat{x}_{val} - c) + \epsilon_{err}
\end{align}

\textit{Gradient Component:} By the chain rule, the gradient of the output is the product of the local derivative and the input gradient: $ \nabla y = \sigma'(x) \cdot \nabla x $. In our abstract domain, we compute an affine approximation of the derivative $ \hat{\sigma}' $ based on the output value $ \hat{y}_{val} $. For the Sigmoid function, we use the property $ \sigma'(x) = \sigma(x)(1 - \sigma(x)) $ to approximate the derivative as $ \hat{y}_{val} \otimes (1 - \hat{y}_{val}) $, where $ \otimes $ denotes affine multiplication. The output gradient is then obtained by multiplying this derivative approximation with the input gradient affine form:
\begin{align}
    \hat{y}_{grad} = \hat{\sigma}' \cdot \hat{x}_{grad}
\end{align}
This approach preserves the correlations between the derivative and the gradient, providing a more precise abstraction than simple interval scaling.

\subsection{Gradient Instability}
A key insight from our analysis is the concept of \textit{gradient instability}. We define a neuron $n$ as having an unstable gradient if its gradient interval $[\underline{g}, \overline{g}]$ contains 0:
\begin{align}
    \exists x_1, x_2 \in X_0 \text{ s.t. } \text{sign}\left(\frac{\partial f}{\partial n}(x_1)\right) \neq \text{sign}\left(\frac{\partial f}{\partial n}(x_2)\right)
\end{align}

Geometrically, this means that the function is non-monotonic w.r.t. neuron $n$ over the input region. When this happens, the assumption of local linearity breaks down and linear relaxations- such as those in \texttt{DeepPoly}- become loose. Our approach explicitly tracks these gradient intervals, thus enabling us to identify whenever and wherever the network's behaviour becomes hard to certify linearly.

\section{Contributions}

\subsection{Algorithmic Realization of Abstract Duals}
While the theoretical foundation rests on dual numbers, our practical contribution is the algorithmic realization of these concepts within the \textbf{Affine Arithmetic} domain. We implemented a custom OCaml module \texttt{AbstractDual} that handles the propagation of affine forms.
Crucially, for non-linear activations $\sigma$, we implement a sound linearization that explicitly accounts for the curvature error. As seen in our implementation, the radius of the value component is expanded by a quadratic term derived from the Taylor expansion:
\begin{equation}
    r_{new} = |\sigma'(c)| \cdot r_{old} + \underbrace{0.5 \cdot \max_{z} |\sigma''(z)| \cdot r_{old}^2}_{\text{Linearization Error}}
\end{equation}
This ensures that the abstract transformer remains sound even when the function has significant curvature, a detail often omitted in standard dual number formulations but critical for verification.

\subsection{Gradient-Based Verification Logic}
We formulate a robust verification condition based on the computed gradient bounds. Let $f_c(x)$ be the score of the correct class and $f_j(x)$ be the score of a competing class. The robustness margin is $M(x_0) = f_c(x_0) - f_j(x_0)$.
To certify robustness over $\mathbb{B}_\infty(x_0, \epsilon)$, we compute the global Lipschitz constant $K$ of the margin function $m(x) = f_c(x) - f_j(x)$ using our abstract duals. The verification condition implemented in our tool is:
\begin{equation}
    M(x_0) > \epsilon \cdot \sup_{x \in \mathbb{B}} ||\nabla m(x)||_1
\end{equation}
In our implementation, we conservatively estimate this by summing the worst-case L1 norms of the gradients for the correct and competing classes:
\begin{equation}
    \text{Certified Variation} = \epsilon \cdot \sum_{i} \max(|g_{low}^{(i)}|, |g_{high}^{(i)}|)
\end{equation}
If the initial margin exceeds twice this variation (accounting for the worst-case drop in $f_c$ and rise in $f_j$), the network is provably robust.

\begin{algorithm}
\caption{Robustness Verification via Abstract Duals}
\label{alg:verification}
\begin{algorithmic}[1]
\Require Neural Network $f$, Input $x_0$, Label $y$, Radius $\epsilon$
\Ensure \textbf{True} if robust, \textbf{False} otherwise

\State \textbf{Step 1: Abstract Initialization}
\State $\hat{x}_{val} \gets \text{AffineForm}(x_0 - \epsilon, x_0 + \epsilon)$ \Comment{Input Box}
\State $\hat{x}_{grad} \gets \text{Identity}(n_{in})$ \Comment{Gradient w.r.t inputs}
\State $\mathcal{X} \gets \langle \hat{x}_{val}, \hat{x}_{grad} \rangle$

\State \textbf{Step 2: Forward Propagation}
\For{layer $l$ in $f$}
    \If{$l$ is Linear($W, b$)}
        \State $\hat{x}_{val} \gets W \hat{x}_{val} + b$
        \State $\hat{x}_{grad} \gets W \hat{x}_{grad}$
    \ElsIf{$l$ is Activation($\sigma$)}
        \State $c \gets \text{center}(\hat{x}_{val})$
        \State $r \gets \text{radius}(\hat{x}_{val})$
        \State $\epsilon_{err} \gets 0.5 \cdot \max |\sigma''| \cdot r^2$ \Comment{Curvature Error}
        \State $\hat{x}_{val} \gets \sigma(c) + \sigma'(c)(\hat{x}_{val} - c) + \epsilon_{err}$
        \State $\hat{\sigma}' \gets \text{AffineApprox}(\sigma', \hat{x}_{val})$
        \State $\hat{x}_{grad} \gets \hat{\sigma}' \otimes \hat{x}_{grad}$ \Comment{Affine Multiplication}
    \EndIf
\EndFor

\State \textbf{Step 3: Certification}
\State $K \gets \sup ||\hat{x}_{grad}||_1$ \Comment{Global Lipschitz Bound}
\State $M \gets f(x_0)_y - \max_{j \neq y} f(x_0)_j$ \Comment{Concrete Margin}
\If{$M > \epsilon \cdot K$}
    \State \Return \textbf{True}
\Else
    \State \Return \textbf{False}
\EndIf
\end{algorithmic}
\end{algorithm}

\subsection{Implementation and Stability Analysis}
We give a functional implementation of our verification environment within OCaml, making use of a Monadic approach to cope with the computational graph. The distinct part of our tool is the \textit{Gradient Stability Check}, used to identify the ``wrapping effect'' in real-time. The metric analyzes the intervals of the gradients on all dimensions corresponding to the inputs. If the interval $[\underline{g}, \overline{g}]$ strictly contains zero, it marks a neuron as unstable. This metric, $\mathcal{I}_{unstable}$, highly correlates with failure cases of relational domains such as $\texttt{DeepPoly}$, providing a novel way of choosing verification heuristics.

\section{Formal Soundness of the Abstract Dual Domain}

Here we establish the soundness of the Abstract Dual domain by demonstrating that the forward-mode propagation of abstract dual numbers yields a guaranteed over-approximation of both the output values and the Jacobian of a neural network for a specific input region. Consequently, the Lipschitz constant extracted from the final abstract gradient serves as a verified robustness certificate.

\subsection{Preliminaries}
Let $f : \R^n \to \R^k$ be a feedforward neural network and let $X_0 \subseteq \R^n$ be a centrally symmetric input region. We utilize \textbf{Affine Arithmetic} for abstraction. An affine form $\hat{z}$ represents a set of values parameterized by noise symbols $\epsilon_j \in [-1, 1]$:
\begin{equation}
    \hat{z} = \alpha_0 + \sum_{j=1}^m \alpha_j \epsilon_j.
\end{equation}
The \textbf{concretization} function $\gamma(\cdot)$ maps an affine form to its corresponding set of real values:
\begin{equation}
    \gamma(\hat{z}) \triangleq \left\{ \alpha_0 + \sum_{j=1}^m \alpha_j \epsilon_j \;\middle|\; \forall j: \epsilon_j \in [-1, 1] \right\}.
\end{equation}
This definition extends element-wise to vectors and matrices.

We define the set of all concrete Jacobians reachable over the input region $X_0$ as:
\begin{equation}
    \mathcalD(f, X_0) \triangleq \left\{ J_f(x) \in \R^{k \times n} \mid x \in X_0 \right\}.
\end{equation}
\begin{definition}[Sound Abstract Dual]
An abstract dual number $\mathcalX \triangleq \langle \hat{x}_{val}, \hat{x}_{grad} \rangle$ is \emph{sound} for a function $f$ over $X_0$ if:
\begin{align}
    \forall x \in X_0: \quad & f(x) \in \gamma(\hat{x}_{val}), \label{eq:sound-val} \\
    \forall x \in X_0: \quad & J_f(x) \in \gamma(\hat{x}_{grad}). \label{eq:sound-grad}
\end{align}
\end{definition}

\subsection{Soundness of Linear Layers}

\begin{lemma}[Linear Layer Soundness]
Let $\mathcalX$ be a sound abstract dual for $f$ over $X_0$. Consider a linear layer $L(x) = Wx + b$. The abstract transformer $\mathcalY \triangleq W\mathcalX + b$ is sound for the composition $L \circ f$ over $X_0$.
\end{lemma}

\begin{proof}
The Jacobian of the affine transformation $L$ is constant: $J_L(x) \equiv W$. By the multivariate chain rule, the Jacobian of the composition is:
\begin{equation}
    J_{L \circ f}(x) = J_L(f(x)) \cdot J_f(x) = W \cdot J_f(x).
\end{equation}

The abstract transformer computes the output components as:
\begin{align}
    \hat{y}_{val}  &= W \hat{x}_{val} + b, \\
    \hat{y}_{grad} &= W \hat{x}_{grad}.
\end{align}

\textbf{1. Value Soundness:}
Since affine arithmetic is exact for linear operations, the concretization satisfies:
\begin{equation}
    \gamma(\hat{y}_{val}) = \setof{ Wz + b \mid z \in \gamma(\hat{x}_{val}) }.
\end{equation}
Because $\mathcalX$ is sound, $f(x) \in \gamma(\hat{x}_{val})$ for all $x \in X_0$. Therefore, $L(f(x)) \in \gamma(\hat{y}_{val})$.

\textbf{2. Gradient Soundness:}
By the inductive hypothesis, $J_f(x) \in \gamma(\hat{x}_{grad})$ for all $x \in X_0$.
Since matrix multiplication is a linear operation, the property of affine arithmetic ensures that:
\begin{equation}
    \forall J \in \gamma(\hat{x}_{grad}), \quad W \cdot J \in \gamma(W \hat{x}_{grad}).
\end{equation}
Substituting the concrete Jacobian $J_f(x)$ for $J$, we obtain $W \cdot J_f(x) \in \gamma(\hat{y}_{grad})$. Thus, $\mathcalY$ is sound.
\end{proof}

\subsection{Soundness of Nonlinear Activations}

\begin{lemma}[Activation Function Soundness]
Let $\sigma : \R \to \R$ be a differentiable activation function applied element-wise. The abstract dual transformer, defined using interval bounds on the derivative $\sigma'$, is sound.
\end{lemma}

\begin{proof}
Let $h(x) \triangleq \sigma(f(x))$. By the chain rule applied element-wise, the Jacobian is:
\begin{equation}
    J_h(x) = \text{diag}(\sigma'(f(x))) \cdot J_f(x).
\end{equation}

Let the range of the value abstraction be $[l, u] = \text{Range}(\gamma(\hat{x}_{val}))$.
We compute the interval enclosure of the derivative over this range:
\begin{equation}
    \Sigma' \triangleq \left[ \min_{z \in [l,u]} \sigma'(z),\; \max_{z \in [l,u]} \sigma'(z) \right].
\end{equation}

\textbf{1. Value Soundness:}
Standard abstract interpretation techniques for $\sigma$ construct $\hat{h}_{val}$ such that $\sigma(\gamma(\hat{x}_{val})) \subseteq \gamma(\hat{h}_{val})$.

\textbf{2. Gradient Soundness:}
Since $\hat{x}_{val}$ is sound, $f(x) \in [l, u]$ for all $x \in X_0$. Consequently, the concrete derivative $\sigma'(f(x))$ is contained within the interval $\Sigma'$.
The abstract gradient is updated via interval-affine multiplication:
\begin{equation}
    \hat{h}_{grad} \triangleq \Sigma' \otimes \hat{x}_{grad}.
\end{equation}
This operation is conservative: it accounts for every possible scaling factor in $\Sigma'$ applied to every affine form in $\hat{x}_{grad}$. Therefore:
\begin{equation}
    \text{diag}(\sigma'(f(x))) \cdot J_f(x) \in \gamma(\Sigma' \otimes \hat{x}_{grad}).
\end{equation}
This confirms that $J_h(x) \in \gamma(\hat{h}_{grad})$, completing the proof.
\end{proof}

\subsection{Global Lipschitz Soundness}

\begin{theorem}[Sound Lipschitz Certification]
Let $\hat{y}_{grad}$ be the abstract Jacobian at the network output. The computed constant $K_{comp}$ derived from $\hat{y}_{grad}$ is a sound upper bound on the $L_\infty$-Lipschitz constant of $f$ over $X_0$.
\end{theorem}

\begin{proof}
By induction on the network depth, the final abstract gradient satisfies:
\begin{equation}
    \forall x \in X_0: \quad J_f(x) \in \gamma(\hat{y}_{grad}).
\end{equation}

The local Lipschitz constant of $f$ at $x$ with respect to the $L_\infty$ norm is the induced $\infty$-norm of the Jacobian:
\begin{equation}
    \norm{ J_f(x) }_\infty = \max_{i} \sum_{j} | (J_f(x))_{ij} |.
\end{equation}
The global Lipschitz constant is $K = \sup_{x \in X_0} \norm{ J_f(x) }_\infty$.

Let the $(i,j)$-th entry of the abstract Jacobian matrix $\hat{y}_{grad}$ be the affine form $\hat{g}_{ij} = c_0 + \sum_k c_k \epsilon_k$.
The maximum absolute value of this entry is bounded by its $L_1$ coefficient norm:
\begin{equation}
    \sup \left| \gamma(\hat{g}_{ij}) \right| \leq |c_0| + \sum_k |c_k| \triangleq \mu_{ij}.
\end{equation}
We define the computed constant as the maximum row sum of these upper bounds:
\begin{equation}
    K_{comp} \triangleq \max_{i} \sum_{j} \mu_{ij}.
\end{equation}
By the triangle inequality, for any concrete $J \in \gamma(\hat{y}_{grad})$, $\norm{J}_\infty \leq K_{comp}$. Since $J_f(x) \in \gamma(\hat{y}_{grad})$, it follows that $K \leq K_{comp}$.

Finally, by the Mean Value Theorem, for any $x, x' \in X_0$:
\begin{equation}
    \norm{ f(x) - f(x') }_\infty \leq K_{comp} \cdot \norm{ x - x' }_\infty.
\end{equation}
Thus, $K_{comp}$ is a valid Lipschitz certificate.
\end{proof}

\section{Evaluation}

This section presents an evaluation of our method and contrasts its behavior with the \texttt{DeepPoly} abstraction. Concretely, we focus on the following research questions:
\begin{itemize}
    \renewcommand{\labelitemi}{\tiny$\square$}
    \item \textbf{RQ1 (Effectiveness):} In which settings does global gradient analysis yield tighter robustness certificates than layer-wise relational abstractions?
    \item \textbf{RQ2 (Diagnostic Value):} Is the proposed notion of \emph{gradient instability} predictive of verification failure?
    \item \textbf{RQ3 (Scalability and Smoothness):} How does performance vary with network size and choice of activation function (Sigmoid versus ReLU)?
\end{itemize}

\subsection{Experimental Setup}

All experiments were implemented in \textbf{OCaml} (version 4.12+), building on the \texttt{ocaml-nn} library for neural network primitives. Evaluations were performed on a standard MacBook Pro equipped with an Apple M1 Pro processor.

\paragraph{Benchmarks.}
We consider fully connected classifiers trained on the MNIST dataset. To isolate the effect of network capacity, we evaluate three architectures of increasing width:
\begin{itemize}
    \renewcommand{\labelitemi}{\tiny$\square$}
    \item \textbf{Tiny Net:} 784 inputs $\rightarrow$ 2 hidden units $\rightarrow$ 10 outputs.
    \item \textbf{Small Net:} 784 inputs $\rightarrow$ 5 hidden units $\rightarrow$ 10 outputs.
    \item \textbf{Standard Net:} 784 inputs $\rightarrow$ 10 hidden units $\rightarrow$ 10 outputs.
\end{itemize}
All models were trained using RMSProp. Robustness was evaluated under $L_\infty$ perturbations with radii $\epsilon \in \{0.01, 0.02, 0.03, 0.12\}$.

\subsection{Results and Analysis}

\subsubsection{RQ1: When Gradient Bounds Succeed and DeepPoly Fails}

A central empirical observation is the existence of a narrow regime in which the global gradient bound certifies robustness while \texttt{DeepPoly} does not. These cases arise almost exclusively for the \textbf{Tiny Net} architecture with Sigmoid activations at small perturbation radii.

\begin{table}[h]
\centering
\caption{Cases in which the Gradient Method certifies robustness while \texttt{DeepPoly} fails.}
\label{tab:tiny_net_results}
\begin{tabular}{lccc}
\toprule
\textbf{Network} & \textbf{$\epsilon$} & \textbf{Count} & \textbf{\% of Total} \\
\midrule
Tiny Net & 0.01 & 5 & 16.1\% \\
Tiny Net & 0.02 & 3 & 9.7\% \\
Tiny Net & 0.03 & 2 & 6.5\% \\
Small Net & All & 0 & 0.0\% \\
Standard Net & All & 0 & 0.0\% \\
\bottomrule
\end{tabular}
\end{table}

As shown in Table~\ref{tab:tiny_net_results}, at $\epsilon = 0.01$ the gradient-based method recovers over 16\% of the test instances that \texttt{DeepPoly} fails to verify. These instances correspond to the following MNIST indices:
\begin{itemize}
    \renewcommand{\labelitemi}{\tiny$\square$}
    \item $\epsilon = 0.01$: indices 19, 76, 122, 139, 148 (all with true label 8);
    \item $\epsilon = 0.02$: indices 122, 139, 148;
    \item $\epsilon = 0.03$: indices 139, 148.
\end{itemize}

\paragraph{Case Study: Index 139 (Label 8).}
To better understand this discrepancy, we examine the instance at index 139 with $\epsilon = 0.03$.
\begin{itemize}
    \renewcommand{\labelitemi}{\tiny$\square$}
    \item \textbf{\texttt{DeepPoly}:} \texttt{UNSTABLE}. The forward-propagated margin collapses to an interval with effectively zero width, indicating overlap with a competing class.
    \item \textbf{Gradient Method:} \texttt{ROBUST}.
    \item \textbf{Gradient Stability:} \texttt{NO (480)}, indicating that 480 input dimensions exhibit sign changes in the abstract gradient.
\end{itemize}

Despite substantial gradient instability, the \emph{magnitude} of the gradient remains small. Consequently, the certified variation $\epsilon \cdot \|\nabla f\|_1$ stays below the true margin between the predicted class and its nearest competitor. In contrast, \texttt{DeepPoly}'s layer-wise abstraction appears to accumulate sufficient over-approximation error—consistent with the classical wrapping effect—to lose the certificate.

\paragraph{Conclusion (RQ1).}
Global gradient bounds can outperform \texttt{DeepPoly} in shallow, low-capacity networks, where accumulated abstraction error dominates and local gradient magnitudes remain small. This advantage disappears as depth or width increases: for both the Small and Standard networks, \texttt{DeepPoly} strictly dominates, with no recovered cases.

\subsubsection{RQ2: Gradient Instability as a Predictor of Failure}

We next evaluate whether \emph{gradient instability}—defined as the abstract gradient interval containing zero—serves as a reliable indicator of verification difficulty. At the largest perturbation radius $\epsilon = 0.12$, we observe a \textbf{perfect correlation}: every robustness failure coincides with gradient instability, across all architectures.

From a geometric perspective, instability reflects non-monotonicity of the decision function within the input ball. In such regions, first-order linear approximations become inherently loose, explaining the simultaneous failure of both \texttt{DeepPoly} and the Abstract Dual domain. These results support the use of the instability index $\mathcal{I}_{\text{unstable}}$ as a practical diagnostic tool for identifying inherently hard instances.

\subsubsection{RQ3: Effect of Activation Smoothness}

Finally, we repeat the experiments using ReLU activations. In this setting, \texttt{DeepPoly} produces tighter or equal bounds in \textbf{all} cases. The degradation of the Abstract Dual domain can be traced directly to the discontinuous derivative of ReLU. Any unstable ReLU unit forces the abstract derivative to range over $[0,1]$, dramatically inflating the global Lipschitz constant. By contrast, Sigmoid activations admit smooth, bounded derivatives, enabling significantly tighter gradient abstractions.

Overall, the evaluation shows that \texttt{DeepPoly} remains the most robust general-purpose abstraction. However, the Abstract Dual domain offers a complementary perspective: it is particularly effective for \emph{shallow networks with smooth activations}, where it can bypass the cumulative over-approximation inherent to layer-wise methods. Moreover, gradient instability emerges as a meaningful geometric signal, shedding light on when and why verification is likely to fail.


\section{Related Work}

The landscape of neural network verification is broadly divided into complete methods, which provide exact certification at the cost of scalability, and incomplete methods, which utilize abstract interpretation to achieve efficiency.


Initial efforts in the field prioritized exactness. \citet{katz2017reluplex} developed \texttt{Reluplex}, an SMT-based solver that extends the Simplex algorithm to manage the piecewise linear nature of ReLU networks. In a similar vein, \citet{tjeng2019evaluating} modeled the verification problem using Mixed Integer Linear Programming (\texttt{MIPVerify}) to determine precise adversarial boundaries. Although these approaches guarantee soundness and completeness, their NP-complete complexity renders them intractable for modern, large-scale architectures.

To overcome the scalability bottlenecks of complete verifiers, Abstract Interpretation \cite{cousot_radhia_cousot_1977} has been widely adopted. \citet{Gehr2018AI2SA} introduced \texttt{AI2}, a framework employing zonotopes for bound propagation. This was advanced by \citet{singh_gehr_püschel_vechev_2019} with \texttt{DeepPoly}, which combines intervals with polyhedral constraints to handle non-linearities like Sigmoid and Tanh through linear relaxation. While \texttt{DeepPoly} significantly improves scalability, it suffers from the "wrapping effect," where approximation errors accumulate with network depth. \citet{Weng2018TowardsFC} similarly utilized linear bound propagation to expedite robustness certification.


While the theory of dual numbers is well-established for derivative evaluation \cite{griewank2008evaluating}, its utility in certifying robustness has been limited. Unlike purely geometric abstractions, our method uses Abstract Duals to compute global Lipschitz bounds. This allows us to bypass layer-wise error accumulation in specific regimes, offering a distinct advantage in shallow networks where gradient information remains coherent.


\section{Conclusion}

In this work, we explored the power of using Abstract Dual Numbers to certify the local robustness of neural networks. By lifting automatic differentiation into the abstract domain, we derived a method to compute global Lipschitz bounds that can certify invariance without explicit layer-by-layer geometric propagation.

Our comparative analysis with \texttt{DeepPoly} revealed a nuanced landscape. While \texttt{DeepPoly} remains the superior choice for general-purpose verification due to its ability to handle deep dependencies, our Gradient Method identified a specific ``blind spot'' in the relational approach. In shallow, smooth networks (Tiny Net with Sigmoid), our method successfully certified 16.1\% of cases at $\epsilon=0.01$ that \texttt{DeepPoly} failed to verify. This suggests that for low-depth architectures, the global gradient bound can be tighter than the accumulated over-approximation error of layer-wise abstract interpretation.

Finally, we established a strong correlation between \textit{Gradient Instability} and robustness failure. In high-perturbation regimes, the presence of zero in the gradient interval served as a perfect predictor for the inability to certify robustness, highlighting its value as a diagnostic metric for detecting the breakdown of local linearity.


%%
%% The next two lines define the bibliography style to be used, and
%% the bibliography file.
\bibliographystyle{ACM-Reference-Format}
\bibliography{paper}
\end{document}
